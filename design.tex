\documentclass{article}
\usepackage[utf8]{inputenc}

\usepackage{listings}
\usepackage{color}

\definecolor{dkgreen}{rgb}{0,0.6,0}
\definecolor{gray}{rgb}{0.5,0.5,0.5}
\definecolor{mauve}{rgb}{0.58,0,0.82}

\lstset{frame=tb,
  language=Java,
  aboveskip=3mm,
  belowskip=3mm,
  showstringspaces=false,
  columns=flexible,
  basicstyle={\small\ttfamily},
  numbers=none,
  numberstyle=\tiny\color{gray},
  keywordstyle=\color{blue},
  commentstyle=\color{dkgreen},
  stringstyle=\color{mauve},
  breaklines=true,
  breakatwhitespace=true
  tabsize=3
}


\title{Reversi Online Design}
\author{Patrick Rock}
\date{October 2013}

\begin{document}

\maketitle

\section{Introduction}
Reversi is an old strategy game played on an 8x8 board. We will be implementing
an online version of the game. Our implementation will feature a GUI and a simple
min-max AI. The project will be done in three major releases: server, AI, and GUI.
We will be using Github and Pivotal Tracker to manage the project. Pivotal Tracker
is an online tool that helps manage SCRUM Backlogs.

\section{Game Rules}
Developing a game requires a strong understanding of the rules. These rules will
be implemented by a class, the Game Engine.
Two players compete to have the most pieces on the board. 
The game ends when neither player can move.
The players take turns unless one player can't make a valid move. 
The dark player moves first.
Dark must play on a square with a straight occupied line to another dark piece. There 
must be at least one white piece in between them. Dark captures the white pieces
on the line.

\section{High Level Design}

\section{Low Level Design}

\section{Conclusion} % benefits, assumptions, risks/issues

\end{document}

