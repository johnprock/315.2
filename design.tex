\documentclass{article}
\usepackage[utf8]{inputenc}

\usepackage{listings}
\usepackage{color}

\definecolor{dkgreen}{rgb}{0,0.6,0}
\definecolor{gray}{rgb}{0.5,0.5,0.5}
\definecolor{mauve}{rgb}{0.58,0,0.82}

\lstset{frame=tb,
  language=Java,
  aboveskip=3mm,
  belowskip=3mm,
  showstringspaces=false,
  columns=flexible,
  basicstyle={\small\ttfamily},
  numbers=none,
  numberstyle=\tiny\color{gray},
  keywordstyle=\color{blue},
  commentstyle=\color{dkgreen},
  stringstyle=\color{mauve},
  breaklines=true,
  breakatwhitespace=true
  tabsize=3
}


\title{Reversi Online Design}
\author{Patrick Rock}
\date{October 2013}

\begin{document}

\maketitle

\section{Introduction}
Reversi is an old strategy game played on an 8x8 board. We will be implementing
an online version of the game in Java. Our implementation will feature a GUI and a simple
min-max AI. The project will be done in three major releases: server, AI, and GUI.
We will be using Github and Pivotal Tracker to manage the project. Pivotal Tracker
is an online tool that helps manage SCRUM Backlogs.

\section{Game Rules}
Developing a game requires a strong understanding of the rules. These rules will
be implemented by a class, the Game Engine.
Two players compete to have the most pieces on the board. 
The game ends when neither player can move.
The players take turns unless one player can't make a valid move. 
The dark player moves first.
Dark must play on a square with a straight occupied line to another dark piece. There 
must be at least one white piece in between them. Dark captures the white pieces
on the line.

\section{High Level Design}
The following are classes in our application:
\begin{itemize}
\item Game State   - Represents a valid configuration of the Reversi board
\item Location     - Represents a coordinate on the game board
\item Game Engine  - Handles game flow, changing game state, and AI
\item Parser       - Reads commands in the network protocol 
\item GUI          - Renders interface, will have many subclasses for board, menu, etc...
\item Server       - Controller class
\item Client       - Connects to Server over telnet
\end{itemize}

\section{Low Level Design}

\subsection{Game State}
The reversi board is an 8x8 grid. We will represent the board with an 8x8 matrix, 
with 0 for black and 1 for white. The Game State is a simple wrapper for a matrix
The AI must also build a tree where each node represents a state of the game.    

\subsection{Game Engine}
The purpose of the game engine is to handle the high level flow of the game. The engine
can control human vs human, and human vs AI games. It has public methods move(location) and ai\_move().
The engine should have private methods, is\_valid(state) to check for valid game states, and is\_over(state)
to check for end game states. 

\subsection{Parser}
\subsection{GUI}
\subsection{Server}
\subsection{Client}



\section{Conclusion} % benefits, assumptions, risks/issues

\end{document}

